%\documentclass[12pt]{article}
\documentclass[preprint,floatfix] {revtex4} 
\usepackage[utf8x]{inputenc}
\usepackage[spanish]{babel}
\usepackage{hyperref}
\usepackage[backend=biber, style=apa, citestyle=authoryear]{biblatex}
%\usepackage{apacite}
\addbibresource{referencias.bib}

\usepackage{amsmath}
\usepackage{graphicx}
\usepackage[colorinlistoftodos]{todonotes}
%%paquetes internos
\usepackage{subfigure}
\usepackage{amsmath}
\usepackage{xcolor}
\usepackage{color, soul}
\newcommand{\rvec}{\mathrm {\mathbf {r}}} 
\renewcommand{\thetable}{S\arabic{table}}
%%comandos  \modulo \nametarea \prof
\def \modulo{2: Cŕedito Bancario y Empresarial}
\def \nametarea{Actividad 2}
\def \prof{Mtro. Leandro Otaolaurruchi González}
\def \fentrega{18 de marzo de 2021}


\begin{document}
    \begin{titlepage}
    %%comandos  \modulo \nametarea \prof
    \newcommand{\HRule}{\rule{\linewidth}{0.3mm}} % grosor línea horizontal 

    \center % Center everything on the page

    \textsc{Soluciones Oportunes de Capital Humano en colaboración con la Facultad de Ingeniería de la UNAM}\\[1cm]
    \textsc{\large Diplomado Avanzado en Banca y Finanzas}\\[0.5cm]
    \textsc{\large 2021-2}\\[0.5cm]
    
    \HRule \\[0.4cm]
        { \large \bfseries \nametarea\\[0.3cm]Módulo \modulo}\\[0.4cm] % Title
    \HRule \\[1.5cm]
    
    \begin{minipage}{0.4\textwidth}
        \begin{flushleft} 
            \emph{Alumno:}\\Daniel Martínez Pineda % ps quién más
        \end{flushleft}
    \end{minipage}
    ~
    \begin{minipage}{0.4\textwidth}
        \begin{flushright} 
            \emph{Profesor:} \\\prof % Supervisor's Name
        \end{flushright}
    \end{minipage}\\[2cm]
    
    
    {\large \ifx\fentrega\undefined \today \else \fentrega \fi}\\[2cm] 

    \includegraphics[scale=0.35]{logo-soluciones.png} 
    \vfill % Fill the rest of the page with whitespace

\end{titlepage}
    \section{Planteamiento}
        \begin{itemize}
            \item Enlistar las instituciones de crédito autorizadas y en operación por la SHCP agrupadas por Banca Múltiple y Banca de Desarrollo.
            \item ¿Cuáles son ejemplos de fideicomisos públicos? ¿Algunos otorgan créditos? ¿Si es así, puedes dar un ejemplo?
        \end{itemize}
    \section{Respuestas}
        \subsection{Instituciones de crédito autorizadas}
            En México, existen actualmente 51 instituciones de crédito autorizadas por la Comisión Nacional Bancaria y de Valores, las cuales se agrupan en las que son de Banca Múltiple y las que son de Banca de Desarrollo. 
            
            Estas últimas son Sociedades Nacionales de Crédito y solamente son seis (\cite{cnbv-bd}):
            \begin{enumerate}
                \item Banco Nacional de Comercio Exterior.
                \item Banco Nacional de Obras y Servicios Públicos.
                \item Nacional Financiera.
                \item Banco Nacional del Ejército.
                \item Banco del Ahorro Nacional y SErvicios Financieros.
                \item Sociedad Hipotecaria Federal.
            \end{enumerate}
            
            Las 45 Instituciones restantes, entran en la categoría de Banca de Desarrollo. Todas ellas se muestran a detalle con su número de clave del Padrón de Entidades Supervisadas (PES) en el Cuadro S1.
            \newpage
            \begingroup
                \squeezetable
                \begin{table}[h]
                    \caption {\label{tab:table1} Instituciones de Banca Múltiple autorizadas y en operación$^{\mathrm{a}}$} 
                    \begin{ruledtabular}
                        \begin{tabular}{llllll}
                            Núm. & \multicolumn{2}{c}{} & Núm. & \multicolumn{2}{c}{}    \\ 
                            \cline{2-3} \cline{5-6} 
                                  & Cve. PES & Razón social &   &  Cve. PES   & Razon social   \\    \hline
                            $1$  &  62       & Banco Nacional de México, S.A.  &
                            $24$ &  5167  &   Banco Azteca, S.A.
                            \\       
                            $2$  &  67       & Banco Santander (México), S.A.$^\dag$ & 
                            $25$ &  5201  &   Banco Credit Suisse (Mexico), S.A.
                            \\         
                            $3$  &  70   & HSBC México, S.A. & 
                            $26$ & 5495 & Banco Autofín México$^\dag$
                            \\
                            $4$  & 72    & Scotiabank Inverlat, S.A.$^\dag$           &  
                            $27$ &  5465  &   Barclays Bank Mexico, S.A.
                            \\
                            $5$   & 75    & BBVA Bancomer          &    
                            $28$  & 5501    & Banco Ahorro Famsa$^\dag$
                            \\    
                            $6$  & 79    & Banco Mercantil del Norte, S.A.          &     
                            $29$ & 5519  & INTERCAM Banco, S.A.
                            \\
                            $7$   &  82  & Banco Interacciones, S.A.$^\dag$          &  
                            $30$  &  5523  & ABC Capital, S.A.
                            \\
                            $8$ &   86  &   Banco Inbursa, S.A.$^\dag$    &
                            $31$ &  5535  &   Banco ACTINVER, S.A.
                            \\
                            $9$ &   88  &   Banca Mifel, S.A.$^\dag$    &
                            $32$ &  5554  &   Banco COMPARTAMOS, S.A.
                            \\
                            $10$ &   90  &   Banco Regional de Monterrey$^\dag$    &
                            $33$ &  5555  &   Banco Multiva, S.A.
                            \\
                            $11$ &  91  &   Banco Invex, S.A. &
                            $34$ &  5558  &   UBS Bank Mexico, S.A.                                
                            \\
                            $12$ &  93  &   Banco del Bajío, S.A. &
                            $35$ &  5559  &  Bancoppel, S.A. 
                            \\
                            $13$ &  94  &   BANSI, S.A. &
                            $36$ &  5587  &   Consubanco, S.A. 
                            \\
                            $14$ &  99  &   Banca Afirme, S.A. &
                            $37$ &  5588  &   Banco Walt-Mart de México Adelante, S.A. 
                            \\
                            $15$ &  109  &   Bank of America Mexico, S.A. &
                            $38$ &  5599  &   Volkswagen Bank, S.A. 
                            \\
                            $16$ &  119  &   Banco J.P. Morgan, S.A. &
                            $39$ &  5976  &   Banco Base, S.A. 
                            \\
                            $17$ &  122  &   Banco Ve Por Más, S.A. &
                            $40$ &  7447  &   Banco Pagatodo, S.A. 
                            \\
                            $18$ &  124  &   American Express Mexico, S.A. &
                            $41$ &  7657  &   Banco Forjadores, S.A. 
                            \\
                            $19$ &  126  &   Investa Bank, S.A. &
                            $42$ &  7659  &   Bankaool, S.A. 
                            \\
                            $20$ &  782  &   CIBANCO, S.A. &
                            $43$ &  7684  &   Banco Inmobiliario Mexicano, S.A. 
                            \\
                            $21$ &  1186  &   Bank of Tokio-Mitsubishi, S.A. &
                            $44$ &  7718  &   Fundación Donde Banco, S.A. 
                            \\
                            $22$ &  1598  &   Banco Monex, S.A. &
                            $45$ &  7841  &   Banco BANCREA, S.A. 
                            \\
                            $23$ &  4385  &   Deutsche Bank Mexico, S.A.
                        \end{tabular}
                    \end{ruledtabular}
                    \begin{tabbing}
                        $^{\mathrm{a}}$Ref.~\cite{cnbv-bm}. \hspace{25pt} \= 
                        $^\dag$Institución de Banca Múltiple, Grupo Financiero.
                    \end{tabbing}
                \end{table}
            \endgroup
        \subsection{Fideicomisos públicos}
            El fideicomiso público en México engloba los contratos en los que el Estado mexicano transmite un bien o un derecho público a otra parte con un fin determinado. La propiedad pública pasa del gobierno federal o los ayuntamientos a otros sujetos que tienen la obligación de darle una utilidad concreta. Su propósito, ese entonces auxiliar al Ejecutivo Federal en las atribuciones del Estado para impulsar las áreas prioritarias y estratégicas del desarrollo (\cite{fideicomisos-sarahi}).
            
            Ahora bien, partiendo de su propósito, se pueden listar algunas de sus funcionalidades (\cite{fideicomisopublico-victoria}):
            
            \begin{enumerate}
                \item Asignar determinados recursos y bienes públicos a fines específicos.
                \item Formar una organización distinta con un patrimonio asignado que funcione de forma independiente del ente público. 
                \item Proporcionar soporte legal a los objetos del fideicomiso para que se gestionen de forma independiente y orientados a cumplir su fin. 
            \end{enumerate}
            
            Además en el fideicomiso público se logran identificar diferentes elementos que lo conforman, entre ellos el contrato, elementos personales, fideicomitente, fiduciario, fideicomisario, fines u objetivos y el patrimonio fideicometido (\cite{fideicomisopublico-victoria}).
            
            También con base en sus características iniciales, los fideicomisos se agrupan en:
            \begin{enumerate}
                \item La finalidad con la cual se hayan creado (fideicomiso público de administración, fideicomiso público de inversión, fideicomiso público de garantía).
                \item El modo de financiación (reembolsable, no reembolsable o mixto).
            \end{enumerate}
            
            A manera de ejemplo, se exponen algunos fideicomisos públicos destinados a:
            \begin{itemize}
                \item Financiar obras públicas para viviendas, centros educativos, hospitales.
                \item Financiar desarrollo agrícola a través de la compra de maquinaria y herramientas.
                \item Financiar un programa de becas para estudios superiores en universidades estatales.
                \item La recuperación de infraestructuras en zonas afectadas por desastres naturales (inundaciones, terremotos, huracanes, entre otros).
                \item Financiar programas contra la delincuencia organizada mediante la compra de recursos de equipamiento y armas para las fuerzas del orden.
                \item  Financiar programas de prestación social, como gastos médicos, pago de jubilación, pago de pensiones, etc.
                \item Financiar programas para el fomento de la inversión en zonas deprimidas del país.
            \end{itemize}
            

    \printbibliography
\end{document}