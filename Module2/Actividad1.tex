\documentclass[12pt]{article}

\usepackage[utf8x]{inputenc}
\usepackage[spanish]{babel}
\usepackage{hyperref}
\usepackage[backend=biber, style=apa, citestyle=authoryear]{biblatex}
%\usepackage{apacite}
\addbibresource{referencias.bib}

\usepackage{amsmath}
\usepackage{graphicx}
\usepackage[colorinlistoftodos]{todonotes}
%%comandos  \modulo \nametarea \prof
\def \modulo{2: Cŕedito Bancario y Empresarial}
\def \nametarea{Actividad 1}
\def \prof{Mtro. Leandro Otaolaurruchi González}


\begin{document}
    \begin{titlepage}
    %%comandos  \modulo \nametarea \prof
    \newcommand{\HRule}{\rule{\linewidth}{0.3mm}} % grosor línea horizontal 

    \center % Center everything on the page

    \textsc{Soluciones Oportunes de Capital Humano en colaboración con la Facultad de Ingeniería de la UNAM}\\[1cm]
    \textsc{\large Diplomado Avanzado en Banca y Finanzas}\\[0.5cm]
    \textsc{\large 2021-2}\\[0.5cm]
    
    \HRule \\[0.4cm]
        { \large \bfseries \nametarea\\[0.3cm]Módulo \modulo}\\[0.4cm] % Title
    \HRule \\[1.5cm]
    
    \begin{minipage}{0.4\textwidth}
        \begin{flushleft} 
            \emph{Alumno:}\\Daniel Martínez Pineda % ps quién más
        \end{flushleft}
    \end{minipage}
    ~
    \begin{minipage}{0.4\textwidth}
        \begin{flushright} 
            \emph{Profesor:} \\\prof % Supervisor's Name
        \end{flushright}
    \end{minipage}\\[2cm]
    
    
    {\large \ifx\fentrega\undefined \today \else \fentrega \fi}\\[2cm] 

    \includegraphics[scale=0.35]{logo-soluciones.png} 
    \vfill % Fill the rest of the page with whitespace

\end{titlepage}
    \section{Planteamiento}
        \subsection{Caso 1}
            Me dijo mi tío Luis:
            
            ---La única forma de que yo haga patrimonio en el tiempo es contratando créditos.
            
            Le dije: 
            
            ---No me gusta deber, me da miedo.
            
            A lo que él me respondió:
            
            ---No te preocupes m'ijo, si no pides créditos para comprar tu casa, tu coche, comedor u otro bien, es muy difícil que puedas ahorrar siempre para comprarte cosas. De hecho, si ahorras y te esperas hasta juntar para comprarte un coche, utilizarás el transporte público los siguientes 10 años.
        \subsection{Requisitos}
            \begin{itemize}
                \item Después de leer el caso, dar la opinión del mismo.
                \item En el lenguaje común se usan indistintamente, pero: ¿Cuál es la diferencia entre préstamo y crédito?
            \end{itemize}
    \section{Respuestas}
        \subsection{Opinión}
            En particular, luego de la primera sesión de este módulo, me parece acertada la postura del tió Luis. El crédito es necesario para poder crecer y progresar en la sociedad actual. Con base en el Caso 1, se puede inferir que el tío Luis no debe tener una edad tan avanzada (por su postura ante los créditos) o debe estar instruido un poco en materia de finanzas. En cuanto a su sobrino, es entendible que tenga miedo a deber, pues quizá nunca ha tenido una tarjeta de crédito o sacado alguna hipoteca o préstamo.
            
            Sin embargo, considero que la decisión más sabia que puede tomar el sobrino de Luis es seguir su consejo e incursionar un poco en la literatura financiera acerca de los beneficios que un crédito puede otorgar. 
        \subsection{Diferencia entre préstamo y crédito}
            Aunque suelen usarse como sinónimos, la diferencia radica en que el crédito busca cumplir o ser destinado para un fin en específico (p.e, crédito hipotecario, crédito empresarial) y desde el inicio tiene establecida la tasa de interés y fechas exactas para el pago de éste. En cambio, un préstamo se da en una sola exhibición sin un fin en específico, brindando así la libertad al prestatario de si en un principio pretendía usarlo para emprender un negocio, puede cambiar de parecer y usarlo para remodelar su casa, por ejemplo. 
    \printbibliography
\end{document}